\documentclass[a4paper,12pt]{article} 
\usepackage{times}
\usepackage[top=1in,bottom=1in,left=1in,right=1in,nohead]{geometry}
\usepackage{graphics}
\usepackage[colorlinks=true,allcolors=Blue]{hyperref}
\usepackage[usenames,dvipsnames]{xcolor}

\begin{document}
\renewcommand{\rmdefault}{ptm}
\pagestyle{empty} 

\setlength{\parindent}{0mm} 
\setlength{\parskip}{5mm}

\begin{flushright}
\today
\end{flushright}

\emph{To:}\newline
Dr. Jerzy A. Filar\newline
Editor-in-Chief\newline
Environmental Modeling and Assessment, Springer

\emph{From:}\newline
Dr. Marcus W. Beck\newline
US Environmental Protection Agency\newline
\href{mailto:beck.marcus@epa.gov}{beck.marcus@epa.gov}, +18509342480

Dr. Rebecca R. Murphy\newline
UMCES at Chesapeake Bay Program\newline
\href{mailto:rmurphy@chesapeakebay.net}{rmurphy@chesapeakebay.net}, +14102679837 \vspace{14.5pt}

Enclosed please find our manuscript, entitled ``Comparison of weighted regression and additive models for trend evaluation of water quality in tidal waters'', to be considered as an original research article in Environmental Modeling and Assessment.  The increasing quantity of monitoring data in coastal environments requires the use of statistical methods that leverage the information provided by long-term time series. Last year we published a manuscript in this journal that described the initial adaptation of a weighted regression model to describe water quality trends in tidal waters (Beck and Hagy 2015, doi: \href{http://dx.doi.org/10.1007/s10666-015-9452-8}{10.1007/s10666-015-9452-8}).  An additional method that uses Generalized Additive Models has also been used to describe trends in Chesapeake Bay (Harding et al. 2016, doi: \href{http://dx.doi.org/10.1007/s12237-015-0023-7}{10.1007/s12237-015-0023-7}).  Since publication, there has been considerable interest in the relative abilities of both methods for trend analysis. Our paper provides a quantitative description of both methods with application to a 29 year time series of chlorophyll data in the Patuxent River Estuary. Accordingly, we feel this information is timely as both methods are currently used by monitoring agencies despite no empirical comparisons.   

Please note that Drs. Jeffrey Chanat and Jeremy Testa have provided informal reviews of an earlier draft.  Further, the co-authors work closely with Drs. Jim Hagy, Bob Hirsch, Jennifer Keisman, and Elgin Perry.  Reviews by either of these individuals would be a conflict of interest.  Finally, we have compiled the manuscript using a \LaTeX\ document preparation system with the Springer template and can provide source files if needed. We greatly appreciate the opportunity to publish our work in Environmental Modeling and Assessment.\vspace{0.5in}

\hspace{4.5in}Respectfully,

\hspace{4.5in}Marcus W. Beck

\end{document}